\chapter{Built-in Commands}
Built-in commands have special uses. The syntax is

\begin{lstlisting}
CommmandName arg1, arg2, ...
\end{lstlisting}
The arguments are separated by commas. Commands should have at least one argument.
\section{dofile}
"dofile" command is used to execute a file, which takes one or two argument. The first argument is the path of the file. The second argument is the encoding of the file. If the second argument is omitted, then the encoding is UTF-8.
\section{import}
"import" command is used to load the module of Cygni. The modules should be put into the 'lib' folder, which is inside the folder of the interpreter.
\section{loaddll}
TO DO:
\section{scope}
The "scope" command takes one argument. The value of the argument can be either 'display' or 'clear'. If the argument is 'display', this command will print every variable in the global scope. Otherwise, this command will clear everything in the global scope.
\section{delete}
The "delete" command is used to remove variable in the global scope. The arguments are the name of the variables.
