\chapter{Functions}
\section{Declaring Functions}
In Cygni, the definition of a function consists of def keyword, 
followed by the function name, 
followed by a list of input arguments enclosed in parentheses, 
followed by the body of the function enclosed in curly braces:

\begin{lstlisting}
def FunctionName([parameters]) {
	# Function Body
}
\end{lstlisting}
The function will return the last value of the function body without the return statement. If you write a function without "return" statement, the execution of the function will return null value.

In Cygni, you can wrap C\# methods as Cygni functions, which are called the native functions. The usage of native functions is the same as Cygni functions.
\section{Invoking Functions}
\begin{lstlisting}
def square(x) {
	return x * x
}

square(15)
\end{lstlisting}

\section{Recursion}
A classical example is the computation of factorial.
\begin{lstlisting}
def fact(n) {
	if n == 0 {
		return 1
	} else {
		return n * fact(n - 1)
	}
}
fact(10)
\end{lstlisting}
\section{Higher-order Functions}
The function can be passed as a parameter. 
\begin{lstlisting}
def mul(x){
	return x * y
}

def mul2(f, x){
	return f(x, 2)
}
\end{lstlisting}
