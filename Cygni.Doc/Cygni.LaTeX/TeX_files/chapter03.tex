\chapter{Functions}
\section{Declaring Functions}
In Cygni, the definition of a function consists of def keyword, 
followed by the function name, 
followed by a list of input arguments enclosed in parentheses, 
followed by the body of the function enclosed in curly braces:

\begin{lstlisting}
def FunctionName([parameters]) {
	# Function Body
}
\end{lstlisting}
The function will return the last value of the function body without the return statement. Generally, it is better to write the return statement to make the code more readable. If you don't want to return any value, just write "return null".

In Cygni, you can wrap C\# methods as Cygni functions, which are called the native functions. The usage of native functions is the same as Cygni functions.
\section{Invoking Functions}
\begin{lstlisting}
def square(x) {
	return x * x
}

square(15)
\end{lstlisting}

\section{Recursion}
A classical example is the computation of factorial.
\begin{lstlisting}
def fact(n) {
	if n == 0 {
		return 1
	} else {
		return n * fact(n - 1)
	}
}
fact(10)
\end{lstlisting}

\section{Collection}
\subsection{List}
The syntax of initializing a list is the same as Python. You can put objects from various types into a list.

A list can be indexed by a non-negative integer. The index starts from zero.
\begin{lstlisting}
list1 = [1, 2, 3, 4, 5]
list2 = [18, false, "Judy"]
list1[0] = 789
\end{lstlisting}

\subsection{HashTable}
Using function 'hashtable' to initialize a hash table by key-value pairs. The key can only be integer, boolean and string.
A hash table can be indexed by the key.
\begin{lstlisting}
ht1 = hashtable("key1",123,"key2",789)
ht1["key1"]
\end{lstlisting}

\section{Structure Array}
The stucture array type is inspired by the Matlab/Octave. You can get element from a struct by the field or by the integer index.
\begin{lstlisting}
s1 = struct("name","Judy","age",16)
s1.name # The same as s1[0]
s1.age
\end{lstlisting}

\section{Function}
\subsection{Cygni Function}
Cygni function should be initialized by the 'def' statement. The function can be passed as a parameter. 
\begin{lstlisting}
def mul(x){
	return x * y
}

def mul2(f, x){
	return f(x, 2)
}
\end{lstlisting}

\subsection{Native Function}
Native function is imported from C\#.

\section{Class}
Cygni class can be initialized by the 'class' statement. It supports inheritance. There are some built-in functions to be overrided as followings.
\begin{itemize}
	\item \_\_INIT\_\_: Constructor for the class. The default constructor is a non-arg constructor.
	\item \_\_TOSTRING\_\_: Output the class instance as a string.
	\item \_\_ADD\_\_: override '+' operator.
	\item \_\_SUBTRACT\_\_: override '-' operator.
	\item \_\_MULTIPLY\_\_: override '*' operator.
	\item \_\_DIVIDE\_\_: override '/' operator.
	\item \_\_MODULO\_\_: override '\%' operator.
	\item \_\_POWER\_\_: override '\^' operator.
	\item \_\_COMPARETO\_\_: return a integer to indicate the comparision result. This function will override the '>', '<', '>=', '<=' operators.
	\item \_\_INDEXER\_\_: This function takes a list as indexes, and return an element.
\end{itemize}

\section{User Data}
User data is a wrapper for the C\# data type.
