\chapter{Classes}
The data and functions within a class are known as the class's members.
The following example displays
\section{Data Members}
Data members are those members that contain data for the class.
\begin{lstlisting}
\end{lstlisting}

\section{Function Members}
Functions members are functions defined within the class. The 'function members' includes not only functions, but also constructors, operators, indexers, etc. An example is as followings:
\begin{lstlisting}
class position{
	def __INIT__(nx, ny){
		this.x = nx
		this.y = ny
	}

	def move(nx, ny){
		this.x = nx
		this.y = ny
	}

	def __TOSTRING__(){
		printf("({0}, {1})", this.x, this.y)
	}

}
p1 = position(10,20)
print(p1)
p1.move(35, 47)
print(p1)
\end{lstlisting}	


\subsection{Constructors}
The default constructor is a non-arg function, which will only initialize the already-exist fields. You can overload the constructor by writing a function which is named as "\_\_INIT\_\_".

\begin{lstlisting}
class MyClass {
	def __INIT__([arguments]) {
		# initialzze the class
	}
}
\end{lstlisting}

In the example "position", the constructor of class "position" initialize two fields "x" and "y".

\subsection{Implementation Inheritance}
If you want to declare that a class derives from another class, use the following syntax:

\begin{lstlisting}
class myDerivedClass: myBaseClass {
	# functions and data members here
}
\end{lstlisting} 

The Cygni class does not support multiple inheritance. Namely, a class can only inherit from one class.

\begin{lstlisting}
class person {
	def __INIT__(name, age) {
		this.name = name
		this.age = age
	}

	def say() {
		print('I am a person!')
	}
}

class employee : person {
	def __INIT__(name, age, salary) {
		base.__INIT__(name, age)
		this.salary = salary
	}

	def say() {
		print('I am an employee!')
	}
}
\end{lstlisting}


\subsection{Hiding Function Members}
If a function with the same name is declared in both base and derived classs, the function in the derived class will hide the one in the base class.

% \begin{lstlisting}
% \end{lstlisting}
There are some built-in functions that can be overwrited.
\begin{itemize}
	\item \_\_INIT\_\_: Constructor for the class. The default constructor is a non-arg constructor.
	\item \_\_TOSTRING\_\_: Output the class instance as a string.
	\item \_\_ADD\_\_: override '+' operator.
	\item \_\_SUB\_\_: override '-' operator.
	\item \_\_MUL\_\_: override '*' operator.
	\item \_\_DIV\_\_: override '/' operator.
	\item \_\_MOD\_\_: override '\%' operator.
	\item \_\_POW\_\_: override '\^' operator.
	\item \_\_CMP\_\_: return a integer to indicate the comparision result. This function will override the '>', '<', '>=', '<=' operators.
	% \item \_\_INDEXER\_\_: This function takes a list as indexes, and return an element.
\end{itemize}


\subsection{Calling Base Versions of Functions}
The variable "base" is its parent class, you can call functions in the base class by it. 
