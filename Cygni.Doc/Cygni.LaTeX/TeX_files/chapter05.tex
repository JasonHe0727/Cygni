\chapter{Classes}
The data and functions within a class are known as the class's members.
The following example displays
\section{Data Members}
Data members are those members that contain data for the class.
\begin{lstlisting}
\end{lstlisting}

\section{Function Members}
Functions members are functions defined within the class. The 'function members' includes not only functions, but also constructors, operators, indexers, etc. An example is as followings:
% \begin{lstlisting}
% class student {
% 	def __INIT__(ID, name, age) {
% 		this.ID = ID
% 		this.name = name
% 		this.age = age
% 	}
% 
% 	def sayMyName() {
% 		printf('My name is {0}.', this.name)
% 	}
% 
% 	def showMyID() {
% 		printf('My ID is {0}.', this.ID)
% 	}
% 
% 	def __TOSTRING__() {
% 		return '(ID: {0}, name: {1}, age: {2})'.format(this.ID, this.name, this.age)
% 	}
% }
% \end{lstlisting}

\begin{lstlisting}
class position{
	def __INIT__(nx, ny){
		this.x = nx
		this.y = ny
	}
	
	def move(nx, ny){
		this.x = nx
		this.y = ny
	}
	
	def __TOSTRING__(){
		printf("({0}, {1})", this.x, this.y)
	}
	
}
p1 = position(10,20)
print(p1)
p1.move(35, 47)
print(p1)
\end{lstlisting}	


\subsection{Constructors}
The default constructor is a non-arg function, which will only initialize the already-exist fields. You can overload the constructor by writing a function which is named as "\_\_INIT\_\_".

\begin{lstlisting}
class MyClass {
	def __INIT__([arguments]) {
		# initialzze the class
	}
}
\end{lstlisting}

In the example "position", the constructor of class "position" initialize two fields "x" and "y".

\subsection{Implementation Inheritance}
If you want to declare that a class derives from another class, use the following syntax:

\begin{lstlisting}
class myDerivedClass: myBaseClass {
	# functions and data members here
}
\end{lstlisting} 

The Cygni class does not support multiple inheritance. Namely, a class can only inherit from one class.

\subsection{Hiding Function Members}
If a function with the same name is declared in both base and derived classs, the function in the derived class will hide the one in the base class.

% \begin{lstlisting}
% \end{lstlisting}

\subsection{Calling Base Versions of Functions}
If you want to call functions in the base class, you can use "base" 
