\chapter{Core Cygni}
Ok, if you have some programming experience before, I think maybe you don't have to go through all of the contents. Some small examples may be more helpful. Let's begin.

\section{Your First Cygni Program}
Here is a simple script which writes message to the screen.

\subsection{The Code}
\begin{lstlisting}
print("Hello Cygni!")
\end{lstlisting}

\subsection{Running the Program}
Input the previous code in the interpreter, then you will see the input "Hello Cygni!" in the screen.
\section{Variables}
The variables in Cygni are dynamic, which means they can be any type, such as number, boolean, string, etc.

You declare variables in Cygni by simply assigning values to them:
\begin{lstlisting}
a = 12
b = true
\end{lstlisting}

You may assign one value to several variables:
\begin{lstlisting}
a = b = 89.32
\end{lstlisting}

\section{Variable Scope}
The scope of a variable is the region of code from which variable can be accessed. In Cygni, the scope is determined by the following rules:
\begin{itemize}
	\item A class, which contains fields, has its own scope.
	\item A function, which contains local variables, has its own scope.
	\item If a variable is not defined in a class or a scope, then it is a global variable, which stays in the global scope.
	\item Built-in variables are in the built-in scope, which is immutable in the runtime.
\end{itemize}

The rule of finding variables is: local -> field -> global -> built-in

\section{Types}
See table \ref{predefined_types}.
\begin{table}
	\begin{tabular}{|c|c|c|c|}
		\hline
		Name & CTS Type & Description & Range(Approximate)  \\ 
		\hline
		number & System.Double & 64-bit, double precision floating point & $ \pm 4.9 \times 10 ^ {-324} \sim 1.7 \times 10 ^ {308} $ \\
		boolean & System.Boolean & Represents true or false & true or false \\
		string & System.String & Unicode character string & \\
		\hline
	\end{tabular}
	\label{predefined_types}
	\caption{Predefined Types}
\end{table}

\subsection{The number type}
The number type is actually the double type in C\#.
\subsection{The boolean type}
The boolean type can only be true or false.
\subsection{The string type}
Literals of the string type should be enclosed by '' or "". Backslashes('\\') represents escape sequence. If you write symbol '@' at the beginning of the sentence, the '\\' will lose its meanning.
\begin{lstlisting}
s1 = "a string"
s2 = 'another string'
s3 = @'c:\test.txt' # If literals start with '@', the backslashes will lose its meanning
s4 = 'c:\\test.txt' # The 's4' equals to 's3;
\end{lstlisting}

\section{Flow Control}
\subsection{Conditional Statement: The if Statement}
The syntax of 'if' statement is as followings:
\begin{lstlisting}
if condition {
	# Do something
} else {
	# Do something
}
\end{lstlisting}
The final 'else' statement is not a must. If you have several branches, it is convenient to use 'elif' statement for represent them. 
\begin{lstlisting}
if condition {
	# Branch One 
} elif {
	# Branch Two 
} elif {
	# Branch Three
} 
\end{lstlisting}

\subsection{Loops}
\subsubsection{The for loop} 
\subsubsection{The foreach loop}
\subsubsection{The while loop}
\section{Factorial}
This is a very traditional example. 


\begin{lstlisting}
def fact(n){
	if n == 0 { 1 }
	else { n * fact(n - 1) }
}
fact(10)
=> 3628800
\end{lstlisting}

\section{Position}
Have a look at another example.
\begin{lstlisting}
class Position{
	def __INIT__(nx, ny){
		this.x = nx
		this.y = ny
	}
	
	def move(nx, ny){
		this.x = nx
		this.y = ny
	}
	
	def __TOSTRING__(){
		printf("({0}, {1})", this.x, this.y)
	}
	
}

p1 = Position(10,20)
print(p1)
p1.move(35,47)
print(p1)
\end{lstlisting}	

Hope you have got a feel about Cygni!
