\chapter{Structures}
The structure is a container, used to organize simple data. The syntax is
\begin{lstlisting}
struct('field1', value1, 'field2', value2, ...)
\end{lstlisting}
The struct can be initialized by "struct" function. The arguments must be in pairs, and each pair consists of a field and a value. Note that the field must be string.

Here is an example:
\begin{lstlisting}
Susan = struct('name', 'Susan', 'age', 31)
=>  struct: {
		name: "Susan"
		age: 31
}
Susan.name
Susan.age
\end{lstlisting}
The function can return a struct, namely, you can use it as a stucture constructor.

\begin{lstlisting}
def complex(real, imaginary) {
	return struct('real', real,
	 'imaginary', imaginary)
}
a = complex(10, 20)
\end{lstlisting}

Here are some suggestions. Generally, structures perform better if the number of fields is fairly small, and they have difficult in containing complicated logical. They are suitable for simple data organization. If the logical of the data is more complicated, there exist some better choices, such as classes. 
